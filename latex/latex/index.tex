\begin{DoxyAuthor}{Author}
Goncalo Moniz, Joao Oliveira, Diogo Leao
\end{DoxyAuthor}
The aim of this assignment is to learn how to implement a set of cooperative real-\/time tasks in Zephyr. Replicating the typical structure of embedded software, a mix of periodic and sporadic tasks will be considered.

The system to implement does a basic processing of an analog signal. It reads the input voltage from an analog sensor, digitally filters the signal and outputs it.

. Input sensor\+: Emulated by a 10 kO potentiometer, supplied by the Dev\+Kit 3 V supply (VDD).

. Digital filter\+: moving average filter, with a window size of 10 samples. Removes the outliers (10\% or high deviation from average) and computes the average of the remaining samples.

. Output\+: pwm signal applied to one of the Dev\+Kit leds.

The system shall be structured with at least three tasks, matching the basic processing blocks, namely one task for acquiring the sample, one for filtering and the other to output the signal. The sampling task is periodic, while the other two are sporadic, being activated when new data is available. 